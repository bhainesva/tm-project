% This is "sig-alternate.tex" V2.1 April 2013
% This file should be compiled with V2.5 of "sig-alternate.cls" May 2012
%
% This example file demonstrates the use of the 'sig-alternate.cls'
% V2.5 LaTeX2e document class file. It is for those submitting
% articles to ACM Conference Proceedings WHO DO NOT WISH TO
% STRICTLY ADHERE TO THE SIGS (PUBS-BOARD-ENDORSED) STYLE.
% The 'sig-alternate.cls' file will produce a similar-looking,
% albeit, 'tighter' paper resulting in, invariably, fewer pages.
%
% ----------------------------------------------------------------------------------------------------------------
% This .tex file (and associated .cls V2.5) produces:
%       1) The Permission Statement
%       2) The Conference (location) Info information
%       3) The Copyright Line with ACM data
%       4) NO page numbers
%
% as against the acm_proc_article-sp.cls file which
% DOES NOT produce 1) thru' 3) above.
%
% Using 'sig-alternate.cls' you have control, however, from within
% the source .tex file, over both the CopyrightYear
% (defaulted to 200X) and the ACM Copyright Data
% (defaulted to X-XXXXX-XX-X/XX/XX).
% e.g.
% \CopyrightYear{2007} will cause 2007 to appear in the copyright line.
% \crdata{0-12345-67-8/90/12} will cause 0-12345-67-8/90/12 to appear in the copyright line.
%
% ---------------------------------------------------------------------------------------------------------------
% This .tex source is an example which *does* use
% the .bib file (from which the .bbl file % is produced).
% REMEMBER HOWEVER: After having produced the .bbl file,
% and prior to final submission, you *NEED* to 'insert'
% your .bbl file into your source .tex file so as to provide
% ONE 'self-contained' source file.
%
% ================= IF YOU HAVE QUESTIONS =======================
% Questions regarding the SIGS styles, SIGS policies and
% procedures, Conferences etc. should be sent to
% Adrienne Griscti (griscti@acm.org)
%
% Technical questions _only_ to
% Gerald Murray (murray@hq.acm.org)
% ===============================================================
%
% For tracking purposes - this is V2.0 - May 2012

\documentclass{sig-alternate-05-2015}


\begin{document}

% Copyright
\setcopyright{acmcopyright}
%\setcopyright{acmlicensed}
%\setcopyright{rightsretained}
%\setcopyright{usgov}
%\setcopyright{usgovmixed}
%\setcopyright{cagov}
%\setcopyright{cagovmixed}


% DOI
\doi{10.475/123_4}

% ISBN
\isbn{123-4567-24-567/08/06}

%Conference
\conferenceinfo{PLDI '13}{June 16--19, 2013, Seattle, WA, USA}

\acmPrice{\$15.00}

%
% --- Author Metadata here ---
\conferenceinfo{WOODSTOCK}{'97 El Paso, Texas USA}
%\CopyrightYear{2007} % Allows default copyright year (20XX) to be over-ridden - IF NEED BE.
%\crdata{0-12345-67-8/90/01}  % Allows default copyright data (0-89791-88-6/97/05) to be over-ridden - IF NEED BE.
% --- End of Author Metadata ---

\title{Sentiment Analysis to Predict S\&P 500 Behavior}
\subtitle{[Proposal]}
%
% You need the command \numberofauthors to handle the 'placement
% and alignment' of the authors beneath the title.
%
% For aesthetic reasons, we recommend 'three authors at a time'
% i.e. three 'name/affiliation blocks' be placed beneath the title.
%
% NOTE: You are NOT restricted in how many 'rows' of
% "name/affiliations" may appear. We just ask that you restrict
% the number of 'columns' to three.
%
% Because of the available 'opening page real-estate'
% we ask you to refrain from putting more than six authors
% (two rows with three columns) beneath the article title.
% More than six makes the first-page appear very cluttered indeed.
%
% Use the \alignauthor commands to handle the names
% and affiliations for an 'aesthetic maximum' of six authors.
% Add names, affiliations, addresses for
% the seventh etc. author(s) as the argument for the
% \additionalauthors command.
% These 'additional authors' will be output/set for you
% without further effort on your part as the last section in
% the body of your article BEFORE References or any Appendices.

\numberofauthors{3} %  in this sample file, there are a *total*
% of EIGHT authors. SIX appear on the 'first-page' (for formatting
% reasons) and the remaining two appear in the \additionalauthors section.
%
\author{
% You can go ahead and credit any number of authors here,
% e.g. one 'row of three' or two rows (consisting of one row of three
% and a second row of one, two or three).
%
% The command \alignauthor (no curly braces needed) should
% precede each author name, affiliation/snail-mail address and
% e-mail address. Additionally, tag each line of
% affiliation/address with \affaddr, and tag the
% e-mail address with \email.
%
% 1st. author
\alignauthor
Aditya Bindra\\
       \affaddr{University of Virginia}\\
       \affaddr{Charlottesville, Virginia}\\
       \email{ab4es@virginia.edu}
% 2nd. author
\alignauthor
Paul Cherian\\
       \affaddr{University of Virginia}\\
       \affaddr{Charlottesville, Virginia}\\
       \email{pmc6dn@virginia.edu}
% 3rd. author
\alignauthor
Ben Haines\\
       \affaddr{University of Virginia}\\
       \affaddr{Charlottesville, Virginia}\\
       \email{bmh5wx@virginia.edu}
}
% There's nothing stopping you putting the seventh, eighth, etc.
% author on the opening page (as the 'third row') but we ask,
% for aesthetic reasons that you place these 'additional authors'
% in the \additional authors block, viz.
\date{30 July 1999}
% Just remember to make sure that the TOTAL number of authors
% is the number that will appear on the first page PLUS the
% number that will appear in the \additionalauthors section.

\maketitle
\section{Problem Introduction}
The challenge we have decided to address in our project is
how to use publicly available text based information sources
produced by the United States government in order to predict the behaviors
of financial markets. In particular, our hope is to apply
sentiment analysis techniques to the minutes of the Federal
Open Market Committee (FMOC) in order to identify correlations with
the behaviour of the S\&P 500. The FOMC meets eight times yearly 
and publishes approximately twenty pages of minutes for each meeting.
These minutes discuss the state of the economy and detail a plan
for the monetary policy in the time until the next meeting. The 
frequency of their publication, their scope, and their source 
of a key economic organization leads us to believe that these
documents in particular would be a good source of potential 
insight.

Financial markets are an area where standard
predictive techniques have often been unsuccessful at achieving
accurate results. Stock prices in particular are notorious
for being difficult to forecast. The ability to make accurate 
predictions about economic conditions provides an immediate 
financial incentive for research into this area and an entire 
industry exists whose primary goal is the development of 
techniques to best predict market behaviour. Beyond the pursuit of
profit, having a warning period before downturns could 
also potentially allow for protective or even preventative 
actions to be taken. 

Our particular approach to the problem, and the contribution we 
believe separates us from previous similar work, is our combination
of topic modeling and sentiment analysis into a single pipeline. Our
goal is to automatically parse the document to identify different
topics such as the energy or health care markets. Applying
sentiment analysis to these topics individually will allow us to create
predictions for each of the sectors tracked by the S\&P 500. With this we can long or short the corresponding S\&P 500 sector through publicly traded ETF Sector SPDRs.

Moreover, if we are able to generate a strategy which is able to outperform its index benchmark, the S\&P 500, significantly, our argument that our unique pipeline of performing sentiment analysis after topic modeling is the differentiating factor that many other sentiment analysis quantitative trading strategies have been lacking. Additionally, as we are applying a more granular approach by correlating sentiment to individual S\&P 500 sectors, we are developing a beta-neutral strategy. That is, our strategy will hopefully only outperform the S\&P 500 or track it very closely. This will then result in a $\beta_{S\&P 500}$ very close to 1. This serves as another important distinction which we believe will help us more accurately predict future index performance. 

\section{Related Work}
As mentioned in the previous section, the allure of financial gains
has inspired a large amount of research into the topic of S\&P predictions
in general and sentiment analysis in particular. 

One popular source of data for sentiment analysis is social media. A paper
by Mittal and Goel\cite{mittal:sp} used sentiment analysis on twitter data in combination
with values from the Dow Jones Industrial Average (DJIA) to predict stock prices.
In comparison to our work this paper used a simpler approach to sentiment
analysis and focused primarily on using a neural network to transform the
output from the analysis into concrete stock predictions. The paper was based
on an earlier 2010 work by Bollen et al. that claimed to predict daily DJIA 
fluctuations with 87\% accuracy.\cite{bollen:twitter}

Another common source for data is news articles. For example
Azar in 2009 demonstrated that positive returns could be derived
from text analysis of new sources although a qualification was made
that these returns vanish when trading costs are accounted for.\cite{azar:thesis} In 
contrast to both of these existing works we have chosen an official
government source for data rather than a source that reflects popular opinion.
The benefits of this decision are that our information comes directly from the
entity that decides policy. Downsides are that it limits the amount of data 
available to us and the argument could be made that stock prices are largely
driven by public opinion.

Some previous work has been done on the subject of combining topic modeling and
sentiment analysis by Lu et al.\cite{lu:2011} However, they restrict their application to analyzing
restaurant reviews. We hope that in the process of applying these techniques to our
particular situation we will discover modifications that better suit the algorithms
to this application.
\vfill

\section{Algorithms and Evaluation}
We propose to use a mixture of topic modeling and sentiment analysis to judge the opinions shared by the Fed on each sector. Each Fed Report can be expected to cover opinions about a subset of  general sectors that affect US economy. A particular sector can be generalized to describe a  mix of some topics.  A topic modeling algorithm like LDA\cite{blei:lda} will be used to  the find the topics covered by each sentence or group of sentences. Based on our prior knowledge of the topical mix of each sector, we intend to feed certain seed words for each topic as a  prior for the topic modeling algorithm, to improve accuracy. Once the topic model is extracted for each sentence, we can do the sentiment analysis.
 
We intend to start off with certain seed words for positive/negative sentiment and then grow them with their synonyms from WordNet\cite{kim:dso}. This  also helps us assign sentiments to yet unseen words. Depending on how frequently the synonyms of the unknown word appear in either positive or negative sentiment word lists, the strength of sentiment polarity can be assigned. To combine sentiments in each sentence, we could potentially use named entity tagging to identify the 'subject' and then define a word window around the subject within which the sentiments will be calculated.

Once the sentiment values have been tagged with each topic model, values for each topic will be considered to be a part of the feature space to predict the S\&P Index. The performance of different regression based models (linear, tree-based, SVM etc..)  will be evaluated based on their predictive powers on the respective S\&P 500 sectors. With this, we then intend to long or short the corresponding S\&P 500 sector through the publicly traded Sector SPDR ETFs.

Financial evaluation will be carried out by primarily attempting to identify the predictive power of our technique. As we are employing a near beta-neutral strategy -- a strategy that seeks to only generate alpha over its benchmark, which for our case would be the S\&P 500 -- we seek to generate as high an information ratio as possible. The information ratio is a ratio of a portfolio's returns above the returns of a benchmark, the S\&P 500, to the volatility of those returns. In simpler words, the information ratio is a risk-adjusted rate of return over our benchmark. This will be particularly helpful in determining whether our strategy is able to identify sentiment accurately enough to generate additional returns over what would have normally been achieved by the market over this time. On that note, another financial metric we will be paying particular attention to is CAPM (Capital Asset Pricing Model) $\alpha$. $\alpha$ is the active return of a strategy against a market index used as a benchmark. As such, a high $\alpha$ can help verify whether our strategy is able to pinpoint the sectors that will either over- or underperform the benchmark as a whole. A high $\alpha$ would indicate that we our sentiment analysis on each of our topic models is correctly correlating sentiment to sector returns. 
\vfill


%
% The code below should be generated by the tool at
% http://dl.acm.org/ccs.cfm
% Please copy and paste the code instead of the example below. 
%

%
% End generated code
%

%
%  Use this command to print the description
%
%\printccsdesc

%\keywords{ACM proceedings; \LaTeX; text tagging}

%\end{document}  % This is where a 'short' article might terminate

%ACKNOWLEDGMENTS are optional

%
% The following two commands are all you need in the
% initial runs of your .tex file to
% produce the bibliography for the citations in your paper.
\bibliographystyle{plain}
\bibliography{tmbib}  % sigproc.bib is the name of the Bibliography in this case
% You must have a proper ".bib" file
%  and remember to run:
% latex bibtex latex latex
% to resolve all references
%
% ACM needs 'a single self-contained file'!
%
%APPENDICES are optional
\balancecolumns
%Appendix A
\balancecolumns % GM June 2007
% That's all folks!
\end{document}
